\documentclass[10pt, letterpaper]{article}

\oddsidemargin -0.25in
\evensidemargin -0.25in
\marginparwidth 20pt
\marginparsep 10pt
\topmargin -0.46in
\headsep 0pt
\textheight 9.5in
\textwidth 7.0in

% some very useful LaTeX packages include:
\usepackage{cite}
\usepackage{graphicx}
\usepackage{subfigure}
\usepackage{amsmath}
\usepackage{amssymb}
\usepackage{mathtools}
\usepackage{bm}
\usepackage{epsfig}
\usepackage{times}
\usepackage{color}
\usepackage[normalem]{ulem}
%\usepackage{url}
\usepackage{array}
\usepackage{multirow}
\usepackage{rotating}
\usepackage{mathabx}
\usepackage{tabularx}
%\usepackage{ulem}		%package to underline (\uline, \uuline, \uwave), strike-through (\sout), and masking the text (\xout)
%\usepackage{yhmath}		%package for bigger delimiters and wider accents
%\usepackage{marvosym}	%package that provides various new symbols. For documentation see 106henlich_MarVoSym.pdf in refs\miscellaneous
%\usepackage{../latexStyles/clrscode_mod}
\usepackage[]{authblk}
\usepackage{braket}
\usepackage[inline]{enumitem}

\usepackage[printwatermark]{xwatermark}
\usepackage{xcolor}
\usepackage{transparent}
%\newwatermark[allpages,color=gray!10,angle=0,scale=5,xpos=0,ypos=0]{SAMAR}
\newwatermark[allpages,angle=0,scale=2,xpos=0,ypos=0]{\transparent{0.08}}

% "box" symbols at end of proofs
\def\IEEEQEDclosed{\mbox{\rule[0pt]{1.3ex}{1.3ex}}} % for a filled box
% for an open box instead that will just fit around a closed one
\def\IEEEQEDopen{{\setlength{\fboxsep}{0pt}\setlength{\fboxrule}{0.2pt}\fbox{\rule[0pt]{0pt}{1.3ex}\rule[0pt]{1.3ex}{0pt}}}}

\def\@begintheorem#1#2{\par\bgroup{\scshape #1\ #2. }\it\ignorespaces}
\def\@opargbegintheorem#1#2#3{\par\bgroup%
   {\scshape #1\ #2\ ({\upshape #3}). }\it\ignorespaces}
\def\@endtheorem{\egroup}

\def\proof{\par{\it Proof:} \ignorespaces}
\def\endproof{\hspace*{\fill}~\IEEEQEDclosed\par}

\newtheorem{theorem}{Theorem}
\newtheorem{pavikc}{\textbf{Corollary}}
\newtheorem{pavikl}{\textbf{Lemma}}
\newtheorem{pavikt}{\textbf{Theorem}}
\newtheorem{pavikd}{\textbf{Definition}}
\newtheorem{pavikp}{\textbf{Proposition}}
\newcommand{\argmin}{\operatornamewithlimits{argmin}}
\newcommand{\argmax}{\operatornamewithlimits{argmax}}

\newcommand{\pd}[2]{\frac{\partial#1}{\partial#2}}
\newcommand{\pdd}[2]{\frac{\partial^2#1}{\partial#2^{2}}}
\newcommand{\dpd}[2]{\dfrac{\partial#1}{\partial#2}}
\newcommand{\dpdd}[2]{\dfrac{\partial^2#1}{\partial#2^{2}}}
\newcommand{\halfrac}{\frac{1}{2}}

\newcommand{\todo}{\textcolor{red}}
\newcommand{\eat}[1]{}

\interdisplaylinepenalty=2500
\renewcommand{\arraystretch}{1.5}

\renewcommand{\vec}[1]{\mathbf{#1}}
%\let\oldhat\hat
%\renewcommand{\hat}[1]{\oldhat{\mathbf{#1}}}

\begin{document}

\title{Quiz - XX}
\author{CSXXX}
%\affil[1]{Affiliation}
\date{}

\maketitle
\vspace{-0.25in}


\textbf{Note}:
\begin{enumerate}
\item Use of cellphones/calculators is strictly prohibited. Switch them off.  
\item For each question, circle at most one option. In case of multiple correct choices, circle the most appropriate one.
\item Multiple/ambiguous choices will not be graded. {\let\thefootnote\relax\footnote{2lSKCThthJoLF}}
\item Correct answer fetches +5 marks and incorrect answer fetches -2 marks.
\item You are free to make any reasonable assumption that you may need to logically answer a question.
\end{enumerate}
\vspace{-0.25in}
\begin{center}
\line(1,0){450}
\end{center}





\begin{enumerate}\item The vertex-connectivity, the edge-connectivity and the chromatic number of the following graph are:
\begin{figure*}[h]
\centering
\end{figure*}\\
Ans:

\begin{enumerate*}\item  4		
	
\item  1		
	
\item  2 \eat{*}
	
\item  5

\item  3		
	
\end{enumerate*}


\item \sout{There are four possible grades in a course with 20 students, viz., A, B, C, D. How may distinct grade distributions are possible? A grade distribution is a tuple $(x_1, x_2, x_3, x_4)$ such that $x_i \in \mathbb{N}, 0 < x_i < 100 , 1 \leq i \leq 4, \Sigma_{i=1}^4 x_i = 100$.} \\
Ans:

\begin{enumerate*}\item  3		
	
\item  2 		
	
\item  5 \eat{*}

\item  1		
	
\item  4		
	
\end{enumerate*}


\item Given two equivalence relations $\sim_1, \sim_2$ on a set $S$, define a relation $\sim_3$ on S as: $a \sim_3 b$ if $a \sim_1 b$ and $a \sim_2 b$. Then $\sim_3$ is\\
Ans:

\begin{enumerate*}\item  3		
	
\item  5 \eat{*}

\item  2 		
	
\item  4		
	
\item  1		
	
\end{enumerate*}


\item Let $\cal{G}$ be the set of all simple undirected finite graphs. Define relation $\sim$ on $\cal{G}$ as $G_1 \sim G_2$ if $G_1$ is homomorphic to $G_2$. The relation $\sim$ is: (i)~reflexive (ii)~symmetric (iii)~transitive\\
Ans:

\begin{enumerate*}\item  1		
	
\item  4 \eat{*}
	
\item  5

\item  2 		
	
\item  3		
	
\end{enumerate*}


\item Define the relation $\sim$ on $\mathbb{R}^2$ as $(x_1, y_1) \sim (x_2, y_2)$ if $max\{|x_1|, |y_1|\} = max\{|x_2|,  |y_2|\}$. Here $|x|$ denotes the absolute value of $x$. What do the equivalence classes of $(\mathbb{R}^2, \sim)$ look like?\\
Ans:

\begin{enumerate*}\item  5 \eat{*}

\item  4		
	
\item  1		
	
\item  3		
	
\item  2 		
	
\end{enumerate*}

\item The number of surjections from the set $\{1, 2, 3, 4, 5 \}$ to the set $\{a, b, c\}$ is\\
Ans:
\begin{enumerate*}
	\item 1		
	\item 2 		
	\item 3		
	\item 4		
	\item 5 \eat{*}
\end{enumerate*}
\item Let $S$ be the set of all $3 \times 3$ matrices with entries from $\mathbb{R}$. Define the relation $\sim$ on $S$ as $A \sim B$ if $B = AR$ for some $R \in S$ with non-zero determinant. The relation $\sim$ is: (i)~reflexive, (ii)~symmetric, (iii)~transitive, (iv)~an equivalence relation?\\
Ans:
\begin{enumerate*}
	\item 1		
	\item 2 		
	\item 3		
	\item 4		
	\item 5 \eat{*}
\end{enumerate*}
\item If there exists a graph homomorphism from $G$ to $H$, then the chromatic numbers of $G$ and $H$ are related as\\
Ans:
\begin{enumerate*}
	\item 1		
	\item 2 		
	\item 3		
	\item 4		
	\item 5 \eat{*}
\end{enumerate*}
\item Define a relation $\sim$ on $\mathbb{R}^2$ as $(a, b) \sim (c, d)$ if the determinant of the matrix $\left [ \begin{matrix} a & c \\ b & d   \end{matrix} \right ]$ is greater than or equal to 0. The relation $\sim$ is: (i)~reflexive, (ii)~symmetric, (iii)~transitive, (iv)~an equivalence relation?\\
Ans:
\begin{enumerate*}
	\item 1		
	\item 2 		
	\item 3		
	\item 4		
	\item 5 \eat{*}
\end{enumerate*}
	


\end{enumerate}

\end{document}

